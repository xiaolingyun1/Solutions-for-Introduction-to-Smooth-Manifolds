\chapter{Smooth Manifolds}
%1-1
\begin{problem}
  Let $ X $ be the set of all points $ (x, y) \in \R^2 $ such that $ y = \pm 1 $, and let $ M $ be the quotient of $ X $ by the equivalence relation generated by $ (x, 1) \sim (x, -1) $ for all $ x \neq 0 $. Show that $ M $ is locally Euclidean and second-countable, but not Hausdorff. (This space is called the \emph{line with two origins}.)
  \begin{proof}
    Let $\pi \colon X \rightarrow M$ be the quotient map, where
    \begin{equation*}
      X = \{(x,y) \in \mathbb{R}^2 : y = \pm 1\}
    \end{equation*}
    and the equivalence relation is generated by $(x,1) \sim (x,-1)$ for all $x \ne 0$.
    Let
    \begin{equation*}
      p = \pi(0,1), \quad q = \pi(0,-1)
    \end{equation*}
    denote the two distinct equivalence classes in $M$ corresponding to the two origins.

    To describe a basis for the topology on \(M\), for any open interval \(W \subseteq \mathbb{R}\), define the following sets:
    \begin{equation*}
      \begin{aligned}
        U_W &= \pi\left( \{(x, \pm 1) : x \in W \} \right), &\text{for } 0 \notin W, \\
        U_W^+ &= \pi\left( \{(x, \pm 1) : x \in W \setminus \{0\} \} \cup \{(0,1)\} \right), & \text{for } 0 \in W, \\
        U_W^- &= \pi\left( \{(x, \pm 1) : x \in W \setminus \{0\} \} \cup \{(0,-1)\} \right), & \text{for } 0 \in W.
      \end{aligned}
    \end{equation*}
    These sets are open in the quotient topology on \(M\), where:
    \begin{itemize}
      \item \(U_W\) is a basic open set in \(M\) when \(W\) does not contain 0;
      \item \(U_W^+\) is a neighborhood of the point \(p = \pi(0,1)\);
      \item \(U_W^-\) is a neighborhood of the point \(q = \pi(0,-1)\).
    \end{itemize}
    We next show that $M$ is second-countable and locally Euclidean but not Hausdorff.
    \begin{itemize}
      \item \textbf{Second-countability:} \\
      Define:
      \begin{equation*}
        \mathcal{B} = \{U_W, U_W^{\pm} \colon W \text{is an open interval with rational endpoints.}\}
      \end{equation*}
      Since there are only countably many open intervals in $\mathbb{R}$ with rational endpoints, the collection $\mathcal{B}$ is countable.

      We claim that $\mathcal{B}$ is a basis for the topology on $M$. Let $U \subseteq M$ be any open set and let $x \in U$. We want to find some $B \in \mathcal{B}$ such that $x \in B \subseteq U$.\\
      There are two cases to consider:
      \begin{itemize}
        \item If $x \in M$ corresponds to a point $\pi(x_0, \pm 1)$ with $x_0 \neq 0$, then $\pi^{-1}(U)$ is an open subset of $X$ containing both $(x_0, 1)$ and $(x_0, -1)$ (since they are identified when $x_0 \ne 0$). Since $X$ inherits the subspace topology from $\mathbb{R}^2$, there exists an open interval $W \ni x_0$ with rational endpoints such that $(W \times \{\pm 1\}) \subseteq \pi^{-1}(U)$. Then $U_W = \pi((W \times \{\pm 1\})) \subseteq U$, and $U_W \in \mathcal{B}$ if $0 \notin W$.
        \item If $x = p = \pi(0, 1)$ or $x = q = \pi(0, -1)$, then $x \in U$ implies $\pi^{-1}(U)$ contains either $(0,1)$ or $(0,-1)$ respectively. Since $\pi^{-1}(U)$ is open in $X$, there exists an open interval $W \ni 0$ such that:
        \begin{itemize}
          \item $(W \setminus \{0\}) \times \{\pm 1\} \subseteq \pi^{-1}(U)$,
          \item and either $(0,1) \in \pi^{-1}(U)$ or $(0,-1) \in \pi^{-1}(U)$.
        \end{itemize}
        Hence, either $U_W^+ \subseteq U$ or $U_W^- \subseteq U$, and such sets are in $\mathcal{B}$ because $W$ has rational endpoints.
      \end{itemize}
      Therefore, $M$ is second-countable.
      \item \textbf{Local Euclidean property:} 
      \begin{itemize}
        \item For $x \notin \{p, q\}$, define a map
        \begin{equation*}
          \varphi \colon U_{\R \setminus \{0\}} \rightarrow \R \setminus \{0\}, \quad \pi(x, \pm 1) \mapsto x
        \end{equation*}
        Clearly $\varphi$ is bijective. Let $V$ be an open subset of $\R \setminus \{0\}$, $\varphi^{-1}(V)$ is open if and only if $\pi^{-1} \circ \varphi^{-1}(V)$ is open in $X$. Since
        \begin{equation*}
          \pi^{-1} \circ \varphi^{-1}(V) = (V \times \{-1\}) \cup (V \times \{1\})
        \end{equation*}
        which is open in $X$, $\varphi^{-1}(V)$ is open in $U_{\R \setminus \{0\}}$. This indicates that $\varphi$ is continuous.

        Let $U \subseteq U_{\R \setminus \{0\}}$ be an open subset of $M$, it means $\pi^{-1}(U)$ is open in $X$. Since 
        \begin{equation*}
          \varphi(U) = \{x \colon (x,1) \in \pi^{-1}(U)\} \cup \{x \colon (x, -1) \in \pi^{-1}(U)\}
        \end{equation*}
        is open in $X$, $\varphi$ yields a homeomorphism. Hence, every point with $x \neq 0$ has a neighborhood homeomorphic to $\mathbb{R} \setminus \{0\}$, which is locally Euclidean.
        \item For $x = p$, define the map
        \[
          \psi_+ \colon U^+_{(-1,1)} \to (-1, 1), \quad 
          \psi_+(\pi(x, \pm 1)) = x, \quad \psi_+(p) = 0.
        \]
        This map is well-defined and bijective. To show that $\psi_+$ is a homeomorphism, it suffices to verify that both $\psi_+$ and its inverse are continuous at $p$ and $0$, respectively.\\
        For any $\varepsilon \in (0,1)$, we have
        \[
          \psi_+^{-1}((-\varepsilon, \varepsilon)) = U^+_{(-\varepsilon, \varepsilon)} \quad \text{and} \quad \psi_+(U^+_{(-\varepsilon, \varepsilon)}) = (-\varepsilon, \varepsilon),
        \]
        which shows that $\psi_+$ is continuous at $p$ and its inverse is continuous at $0$. Therefore, $\psi_+$ is a homeomorphism.
        \item For $x = q$, The proof is identical to Case 2.
      \end{itemize}
      \item \textbf{Not Hausdorff:} \\
      We show that $M$ is not Hausdorff by exhibiting two points that cannot be separated by disjoint open neighborhoods.

      Consider the two points $p = \pi(0,1)$ and $q = \pi(0,-1)$. Suppose for contradiction that there exist disjoint open sets $U$ and $V$ in $M$ such that $p \in U$ and $q \in V$.

      Since the sets $U$ and $V$ are open neighborhoods of $p$ and $q$, respectively, there exist basic open sets $U_W^+ \subseteq U$ and $U_{W'}^- \subseteq V$, where $W$ and $W'$ are open intervals containing $0$.
      
      Let $W'' = W \cap W'$; then $0 \in W''$, so $W'' \setminus \{0\} \neq \emptyset$. Define:
      \begin{equation*}
        A := \pi\left( (W'' \setminus \{0\}) \times \{\pm 1\} \right) = U_{W'' \setminus \{0\}} \subseteq U_W^+ \cap U_{W'}^-.
      \end{equation*}
      Therefore, $U_W^+$ and $U_{W'}^-$ are not disjoint; they always intersect in a nonempty open set.\\
      This contradicts the assumption that $p$ and $q$ can be separated by disjoint open sets. Hence, $M$ is not Hausdorff.
    \end{itemize}
  \end{proof}
\end{problem}

%1-2
\begin{problem}
  Show that a disjoint union of uncountably many copies of $\R$ is locally Euclidean and Hausdorff, but not second-countable.
  \begin{proof}
    Let $ X = \coprod_{\alpha \in A}\R_\alpha $, $ A $ is an uncountable index set. $ U \subseteq X $ is open if and only if $ \forall \alpha \in A$, $ \{x\in \R: (\alpha,x)\in U\}$ is open in $\R$.
    \begin{enumerate}
      \item[1.] $X$ is locally Euclidean. \\
      By definition, $\R_\alpha = \{(\alpha,x): x \in \R\}$. $\forall (\alpha, x) \in \R_\alpha$,  since $\R_\alpha$ is an open subset of $X$(By the definition of topology of $X$) and $\R_\alpha$ is homeomorphic to $\R$, $X$ is locally Euclidean.
      
      \item[2.] $X$ is Hausdorff. \\
      Let $(\alpha,x),(\beta,y) \in X$. if $\alpha \neq \beta$, clearly we have two disjoint open subset $\R_\alpha$ and $\R_\beta$ such that 
      $ (\alpha,x) \in \R_\alpha $ and $ (\beta,x) \in \R_\beta $. if $\alpha = \beta$, since $\R$ is Hausdorff, we can find two disjoint open subset $U,V \in \R$ such that $x \in U $ and $y \in V$. $(\alpha,U),(\beta,V)$ are two disjoint open subset of $X$.
      
      \item[3.] $X$ is not second-countable. \\
      Assume that $X$ is second-countable with its countable basis $\mathcal{B}=\{B_i\}$ and $I$ is an countable index set. Since $\R_\alpha$ is an non-empty open subset of $X$, we can always find $B_i \in \mathcal{B}$ such that $B_i \subseteq \R_\alpha$. By Axiom of Choice, we can define 
      $$ f \colon A \rightarrow I, \quad  \alpha \mapsto i \ \text{ s.t. } \ B_i \subseteq \R_{\alpha}. $$
      Clearly $f$ is injective. This leads to an contradiction, since $A$ is uncountable but $I$ is countable.
    \end{enumerate}
  \end{proof}
\end{problem}

%1-3
\begin{problem}
  A topology space is said to be \emph{$\sigma$-compact} if it can be expressed as a union of countably many compact subspaces. Show that a locally Euclidean Hausdorff space is a topological manifold if and only if it is $\sigma$-compact.
  \begin{proof}
    \begin{enumerate}
      \item[($\Rightarrow$)] Every topological manifold admits a countable basis $\mathcal{B} = \{B_i\}$ of precompact coordinate balls(Lemma 1.10). The collection $\{\overline{B_i} \mid B_i \in \mathcal{B}\}$ implies that the manifold is $\sigma$-compact.
      \item[($\Leftarrow$)] Let $X$ be a locally Euclidean Hausdorff space that is $\sigma$-compact. By definition, there exists a countable family of compact subsets $\{K_n\}_{n \in \N}$ such that $X = \bigcup_{n \in \N} K_n$.
      Since $X$ is locally Euclidean, for each $K_n$, there exists a finite open cover $\{U_{n_i}\}_{i=1}^{k_n}$ where each $(U_{n_i}, \varphi_{n_i})$ is a coordinate chart.
      \begin{enumerate}
        \item[1.] For each $U_{n_i}$, choose a precompact coordinate ball $B_{n_i} \subseteq U_{n_i}$ (possible by local Euclideanness, see Lemma 1.10). The collection $\{B_{n_i}\}$ is countable and covers $X$.
        \item[2.] Each $B_{n_i}$, being homeomorphic to an open ball in $\mathbb{R}^n$, admits a countable basis. A countable union of countable bases remains countable, thus $X$ is second-countable.
      \end{enumerate}
      Therefore, $X$ satisfies all axioms of a topological manifold (locally Euclidean + Hausdorff + second-countable).
    \end{enumerate}
  \end{proof}
\end{problem}

%1-4
\begin{problem}
  Let $M$ be a topological manifold, and let $\mathcal{U}$ be an open cover of $M$.
  \begin{enumerate}
    \item  Assuming that each set in $\mathcal{U}$ intersects only finitely many others, show
    that $\mathcal{U}$ is locally finite.
    \item Give an example to show that the converse to (a) may be false.
    \item Now assume that the sets in $\mathcal{U}$ are precompact in $M$, and prove the converse: if $\mathcal{U}$ is locally finite, then each set in $\mathcal{U}$ intersects only finitely many others.
  \end{enumerate}
  \begin{proof}
    \begin{enumerate}
      \item Omitted as "Easy".
      \item Let $M=\R$, $\mathcal{U} = \{(n, \infty):n \in \N\} \cup \{(-\infty, 1)\}$
\item Assume $\mathcal{U}$ is a locally finite open cover of $M$, and that each set in $\mathcal{U}$ is precompact. Fix $U \in \mathcal{U}$, and define
\[
\mathcal{V} = \{V \in \mathcal{U} \colon V \cap U \neq \emptyset\},
\]
the collection of all elements in $\mathcal{U}$ that intersect $U$.

Since $U$ is precompact, its closure $\overline{U}$ is compact. Because $\mathcal{U}$ is locally finite, for every point $x \in \overline{U}$, there exists an open neighborhood  $V_x$ intersects only finitely many elements of $\mathcal{U}$.

Then $\{V_x\}_{x \in \overline{U}}$ is an open cover of $\overline{U}$ by elements of $\mathcal{U}$, so by compactness, there exists a finite subcover:
\[
\overline{U} \subseteq \bigcup_{i=1}^n V_{x_i}.
\]
Now for each $i = 1, \dots, n$, define
\[
\mathcal{V}_i = \{W \in \mathcal{U} \colon W \cap V_{x_i} \neq \emptyset\}.
\]
Since each $V_{x_i}$ intersects only finitely many elements of $\mathcal{U}$, each $\mathcal{V}_i$ is finite.
Now, take any $V \in \mathcal{V}$. Then $V \cap U \neq \emptyset$, and since $\overline{U} \subseteq \bigcup_{i=1}^n V_{x_i}$, there exists some $i$ such that $V \cap V_{x_i} \neq \emptyset$, implying $V \in \mathcal{V}_i$. Thus,
\[
\mathcal{V} \subseteq \bigcup_{i=1}^n \mathcal{V}_i.
\]
As each $\mathcal{V}_i$ is finite and $n$ is finite, it follows that $\mathcal{V}$ is finite.

Therefore, each $U \in \mathcal{U}$ intersects only finitely many other elements of $\mathcal{U}$.
    \end{enumerate}
  \end{proof}
\end{problem}

%1-5
\begin{problem}
  Suppose $M$ is a locally Euclidean Hausdorff space. Show that $M$ is second
  countable if and only if it is paracompact and has countably many connected
  components. 
  \begin{proof}
    \begin{enumerate}
      \item[($\Rightarrow$)] 
      By Proposition 1.11, second-countable property of topological manifold admits at most countably many connected components. Theorem 1.15 shows that every topological manifold is paracompact.

      \item[($\Leftarrow$)] 
      Suppose $M$ is paracompact and has countably many connected components. It suffices to show that each connected component is second countable, since a countable union of second countable spaces is second countable.

      Let $C$ be a connected component of $M$. Since $M$ is locally Euclidean, there exists a basis of precompact coordinate charts. Let $\mathcal{U}$ be an open cover of $C$ by such charts. By paracompactness, there exists a locally finite refinement $\mathcal{V}$ of $\mathcal{U}$ consisting of precompact coordinate domains.

      To show that $C$ is second countable, we will prove that $\mathcal{V}$ is countable. For this, define an equivalence relation $\sim$ on $\mathcal{V}$: for $U, V \in \mathcal{V}$, declare $U \sim V$ if there exists a finite sequence $U = U_0, U_1, \dots, U_n = V$ in $\mathcal{V}$ such that $U_i \cap U_{i+1} \neq \emptyset$ for all $i$. Denote by $[U]$ the equivalence class of $U$ under this relation.

      We now show that $[U]$ is an open and closed subset of $C$:
      \begin{itemize}
        \item \emph{$[U]$ is open:} $U$ is a union of open set by definition.

        \item \emph{$[U]$ is closed:} Let $x \in C \setminus [U]$. Since $\mathcal{V}$ is an open cover of $C$, there exists $V \in \mathcal{V}$ such that $x \in V$. If $V$ intersected any element of $[U]$, then $V$ would be connected to $U$ via a finite chain of overlapping sets, and hence $x \in [U]$, contradicting $x \in C \setminus [U]$. Therefore, $x$ has an open neighborhood contained in $C \setminus [U]$.

        Since this holds for arbitrary $x \in C \setminus [U]$, we conclude that $C \setminus [U]$ is open, so $[U]$ is closed.
      \end{itemize}

      Since $C$ is connected and $[U]$ is nonempty, open, and closed in $C$, it must be that $[U] = C$. Hence, every element of $\mathcal{V}$ can be connected to $U$ via a finite chain of overlapping sets.

      Now define inductively:
      \[
        \mathcal{V}_1 = \{U\}, \quad \mathcal{V}_{n+1} = \{V \in \mathcal{V} \colon \exists W \in \mathcal{V}_n \text{ with } V \cap W \neq \emptyset \}.
      \]
      Then $\bigcup_{n=1}^\infty \mathcal{V}_n = \mathcal{V}$. By \cref{problem:1-4}, each $\mathcal{V}_n$ is finite. Thus, $\mathcal{V}$ is a countable collection.

      Since $\mathcal{V}$ is a countable open cover of $C$ by coordinate domains, the collection
      \[
      \overline{\mathcal{V}} = \{\overline{V} \colon V \in \mathcal{V}\}
      \]
      covers $C$ with countably many compact subsets, thus $C$ is $\sigma$-compact. By \cref{problem:1-3}, $C$ is second-countable.
      Finally, $M$ is a countable disjoint union of its connected components, each of which is second countable, so $M$ is second-countable.
    \end{enumerate}
  \end{proof}
\end{problem}

%1-6
\begin{problem}
  Let $M$ be a nonempty topological manifold of dimension $n \geq 1$. If $M$ has
  a smooth structure, show that it has uncountably many distinct ones. [Hint:
  first show that for any $s>0$, $F_s(x) = |x|^{s-1}x$ defines a homeomorphism
  from $\mathbb{B}^n$ to itself, which is a diffeomorphism if and only if $s = 1$.]

  \begin{proof}
    We proceed in four steps:

    \begin{enumerate}
      \item[1.] \textbf{Homeomorphism property of $F_s$:} \\
      For any $s>0$, the map $F_s \colon \mathbb{B}^n \rightarrow \mathbb{B}^n$ defined by $F_s(x) = |x|^{s-1}x$ is a homeomorphism.
      \begin{itemize}
        \item If $s \geq 1$, $F_s$ is clearly continuous on $\mathbb{B}^n$.
        \item If $0 < s < 1$, continuity at $x=0$ follows from:
        $$ \lim_{x \to 0}|F_s(x)| = \lim_{x \to 0}|x|^s = 0 = F_s(0). $$
        \item The inverse is $F_{1/s}$, since $F_s \circ F_{1/s} = F_{1/s} \circ F_s = \text{id}_{\mathbb{B}^n}$.
      \end{itemize}

      \item[2.] \textbf{Non-smoothness at origin:} \\
      $F_s$ is a diffeomorphism on $\mathbb{B}^n \setminus \{0\}$ but fails to be smooth at $0$ when $s \neq 1$:
      
\begin{itemize}
        \item For $0 < s < 1$, the derivative at $0$ does not exist:
        $$ \frac{\partial F_s(0)}{\partial x^i} = \lim_{\Delta x^i \to 0} (\Delta x^i)^{s-1}(0,\ldots,1,\ldots,0) \quad \text{(diverges)} $$
        \item For $s>1$, the inverse $F_{1/s}$ has $0<1/s<1$ and thus fails to be smooth at $0$.
      \end{itemize}
      Hence $F_s$ is a diffeomorphism on $\mathbb{B}^n$ if and only if $s=1$.

      \item[3.] \textbf{Constructing a modified atlas:} \\
      Fix a point $p \in M$ and choose a smooth chart $(U, \varphi)$ from the given smooth structure $\mathcal{A}$ on $M$, such that:
      $$     \varphi(U) = \mathbb{B}^n \quad \text{and} \quad \varphi(p) = 0.     $$
      For any $s > 0$, define a new chart $(U, \varphi_s)$ by:
      $$     \varphi_s = F_s \circ \varphi,     $$
      where $F_s(x) = |x|^{s-1}x$ is the homeomorphism from Step 1.
  
      Construct a new atlas $\mathcal{A}_s$ as follows:
      $$     \mathcal{A}_s = \left\{ (U, \varphi_s) \right\} \cup \left\{ (V, \psi) \in \mathcal{A} \colon p \notin V \right\}.     $$
      That is, $\mathcal{A}_s$ consists of:
      
  \begin{itemize}
        \item The single modified chart $(U, \varphi_s)$ centered at $p$,
        \item All charts from the original atlas $\mathcal{A}$ that do not contain $p$.
      \end{itemize}
  
      $\mathcal{A}_s$ \textbf{is a smooth atlas:}
      
  \begin{itemize}
        \item The charts in $\mathcal{A}_s$ cover $M$: every point $q \neq p$ is covered by some chart $(V, \psi)$ in $\mathcal{A}$ with $p \notin V$, and $p$ is covered by $(U, \varphi_s)$.
        \item The charts in $\mathcal{A}_s$ are pairwise compatible:
        \begin{itemize}
          \item For any two charts $(V_1, \psi_1)$ and $(V_2, \psi_2)$ in $\mathcal{A}_s$ not containing $p$, their transition map $\psi_2 \circ \psi_1^{-1}$ is smooth because $\mathcal{A}$ is a smooth atlas.
          \item For $(U, \varphi_s)$ and any $(V, \psi)$ with $p \notin V$, the transition map on $U \cap V$ is:
          $$
          \psi \circ \varphi_s^{-1} = \psi \circ \varphi^{-1} \circ F_{1/s}.
          $$
          This is smooth because $\psi \circ \varphi^{-1}$ is smooth (by compatibility in $\mathcal{A}$) and $F_{1/s}$ is smooth away from $0$.
        \end{itemize}
      \end{itemize}
  
      \item[4.] \textbf{Distinct smooth structures:} \\
      We show that the smooth structures induced by $\mathcal{A}_s$ and $\mathcal{A}_{s'}$ are distinct unless $s = s'$.
  
      
  \begin{itemize}
        \item Suppose $s \neq s'$. Consider the transition map between $(U, \varphi_s)$ and $(U, \varphi_{s'})$:
        $$
        \varphi_{s'} \circ \varphi_s^{-1} = F_{s'} \circ F_{1/s} = F_{s'/s}.
        $$
        This is a diffeomorphism on $\mathbb{B}^n \setminus \{0\}$ but fails to be smooth at $0$ unless $s'/s = 1$ (i.e., $s = s'$), as shown in Step 2.
        \item Thus, $\mathcal{A}_s$ and $\mathcal{A}_{s'}$ are not smoothly compatible unless $s = s'$.
      \end{itemize}
  
      Since there are uncountably many choices for $s > 0$, this yields uncountably many distinct smooth structures on $M$.
    \end{enumerate}
  \end{proof}
\end{problem}

%1-7
\begin{problem}
  Let $N$ denote the north pole $(0,\dots,0,1) \in \Sph^n \subseteq \mathbb{R}^{n+1}$, and let $S$ denote the south pole $(0,\dots,0,-1)$. Define the stereographic projection $\sigma\colon \Sph^n \setminus \{N\} \to \mathbb{R}^n$ by$$\sigma(x^1,\dots,x^{n+1}) = \frac{(x^1,\dots,x^n)}{1 - x^{n+1}}.$$ Let $\tilde{\sigma}(x) = - \sigma(-x)$ for $x \in \Sph^n \setminus \{S\}$.
  \begin{enumerate}
    \item For any $x \in \Sph^n \setminus \{N\}$, show that $\sigma(x) = u$, where $(u,0)$ is the point where the line through $N$ and $x$ intersects the hyperplane $x^{n+1} = 0$. Similarly, show that $\tilde{\sigma}(x)$ is the point where the line through $S$ and $x$ intersects the same hyperplane. (Thus $\tilde{\sigma}$ is called stereographic projection from the south pole.)
    \item Show that $\sigma$ is bijective, with inverse given by$$\sigma^{-1}(u^1,\dots,u^n) = \frac{(2u^1,\dots,2u^n, |u|^2 - 1)}{|u|^2 + 1}.$$
    \item Compute the transition map $\tilde{\sigma} \circ \sigma^{-1}$ and verify that the atlas $\{(\Sph^n \setminus \{N\}, \sigma), (\Sph^n \setminus \{S\}, \tilde{\sigma})\}$ defines a smooth structure on $\Sph^n$. These are called stereographic coordinates.
    \item Show that this smooth structure agrees with the one defined in Example 1.31.
  \end{enumerate}
  \begin{proof}
    \begin{enumerate}
      \item Since $N$, $x$, and $\sigma(x)$ are collinear, there exists $\lambda \in \R$ such that 
      \begin{equation*}
        x = \lambda N + (1-\lambda)\sigma(x).
      \end{equation*}
      Solving for $\lambda$ and $\sigma(x)$ gives:
      \begin{align*}
        \lambda      &= x^{n+1}, \\[0.5em]
        \sigma(x)    &= \dfrac{(x^1, \dots, x^n)}{1 - x^{n+1}}.
      \end{align*}
      The symmetry $\tilde{\sigma}(-x) = -\sigma(x)$ implies $\tilde{\sigma}(x) = -\sigma(-x)$.

      \item Verify that $\sigma \circ \sigma^{-1} = \text{id}_{\R^n}$ and $\sigma^{-1} \circ \sigma = \text{id}_{\Sph^n \setminus \{N\}}$
      \begin{itemize}
        \item For $\sigma \circ \sigma^{-1}$, let $(u^1, \dots, u^n) \in \R^n$, we have
        \begin{align*}
          \sigma \circ \sigma^{-1}(u^1, \dots, u^n) &= \sigma \left( \dfrac{(2u^1, \dots, 2u^n, |u|^2-1)}{|u|^2+1}\right)\\[0.5em]
          &= \dfrac{\left(\frac{2u^1}{|u|^2+1}, \dots, \frac{2u^n}{|u|^2+1}\right)}{1-\frac{|u|^2-1}{|u|^2+1}}\\[0.5em]
          &= (u^1, \dots, u^n).
        \end{align*}
        \item For $\sigma^{-1} \circ \sigma$, let $(x^1, \dots, x^{n+1}) \in \Sph^n \setminus\{N\}$, which means $x^{n+1} \neq 1$ and
        \begin{gather*}
          \left| \sigma(x)\right|^2 = \dfrac{\left(x^1\right)^2 + \dots + \left(x^n\right)^2}{\left(1-x^{n+1}\right)^2} = \dfrac{1+x^{n+1}}{1-x^{n+1}}, \\[0.5em]
          \begin{aligned}
              \sigma^{-1} \circ \sigma(x^1, \dots, x^{n+1}) &= \sigma^{-1}\left(\dfrac{(x^1, \dots, x^n)}{1-x^{n+1}}\right) \\
              &= \dfrac{\left(\frac{2x^1}{1-x^{n+1}}, \dots, \frac{2x^n}{1-x^{n+1}},\frac{|x|^2 - 1}{1-x^{n+1}}\right)}{\frac{|x|^2 + 1}{1-x^{n+1}}} \\
              &= (x^1, \dots, x^{n+1}).
          \end{aligned}
      \end{gather*}
      \end{itemize}
      \item It's sufficient to proof that $\tilde{\sigma} \circ \sigma^{-1}$ and $\sigma \circ \tilde{\sigma}^{-1}$ are smooth on $\R^n \setminus \{0\}$.
      Let $ u = (u^1, \dots, u^n) \in \R^n \setminus \{0\}$, it can be easily verified $\tilde{\sigma}^{-1}(u) = \sigma^{-1}(u)$.
      \begin{align*}
        \tilde{\sigma} \circ \sigma^{-1}(u) = \sigma \circ \tilde{\sigma}^{-1}(u) = \dfrac{u}{|u|^2},
      \end{align*}
      both are smooth on $\R^n \setminus \{0\}$.
      \item We only proof that $\sigma \circ \pi_i^{-1}$ and $\pi_i \circ \sigma^{-1}$ are smooth for $i=1, \dots, n+1$, $\tilde{\sigma}$ is completely the same.
      \begin{itemize}
        \item For $i = n+1$, 
        \begin{itemize}
          \item For transition map $\pi_{n+1} \circ \sigma^{-1} \colon$
          \begin{gather*}
            \sigma(U^+_{n+1} \setminus \{N\}) = \sigma \{x^{n+1} \in (0,1)\} = \{|u|>1 \colon u \in \R^n\}. \\[0.5em]
            \begin{aligned}
                \pi_{n+1} \circ \sigma^{-1}(u) &= \pi_{n+1}\left(\dfrac{(2u^1, \dots, 2u^n, |u|^2-1)}{1+|u|^2}\right) \\
                &= \dfrac{2u}{1+|u|^2}.
            \end{aligned}
        \end{gather*}
          \item For transition map $\sigma \circ \pi_i^{-1} \colon$
          \begin{gather*}
            \pi_{n+1}(U^+_{n+1} \setminus \{N\}) = \pi_{n+1}\{x^{n+1} \in (0,1)\} = \B^n \setminus \{0\}. \\[0.5em]
            \begin{aligned}
                \sigma \circ \pi_{n+1}^{-1}(u) &= \sigma (u^1, \dots, u^n, \sqrt{1-|u|^2}) \\
                &= \dfrac{u}{1-\sqrt{1-|u|^2}}.
            \end{aligned}
        \end{gather*}
        \end{itemize} 
        Both of them are smooth on their domains.
        \item For $i= 1, \dots, n$,
        \begin{itemize}
          \item For transition map $\pi_i \circ \sigma^{-1} \colon$
          \begin{gather*}
            \sigma(U_i^+\setminus \{N\}) = \sigma(U_i^+) = \{u \in \R^n \colon u^i >0\}. \\[0.5em]
            \begin{aligned}
                \pi_i \circ \sigma^{-1}(u) &= \pi_i\left(\dfrac{(2u^1, \dots, 2u^n, |u|^2-1)}{1+|u|^2}\right) \\
                &= \dfrac{(2u^1, \dots, \widehat{2u^i}, \dots, |u|^2-1)}{|u|^2+1}
            \end{aligned}
        \end{gather*}
          \item For transition map $\sigma \circ \pi_i^{-1} \colon$
          \begin{gather*}
            \pi_i(U_i^+\setminus \{N\}) = \pi_i(U_i^+) = \B^n \cap \{u \in \R^n \colon u^i >0\}. \\[0.5em]
            \begin{aligned}
                \sigma \circ \pi_i^{-1}(u) &= \sigma(u^1, \dots, \sqrt{1-|u|^2}, u^i, \dots, u^n) \\
                &= \dfrac{(u^1, \dots, \sqrt{1-|u|^2}, \dots, u^{n-1})}{1-u^n}
            \end{aligned}
        \end{gather*}
        \end{itemize}
      \end{itemize}
      All transition maps are smooth on their domains, confirming compatibility.
    \end{enumerate}
  \end{proof}
\end{problem}

%1-8
\begin{problem}
  By identifying $\R^2$ with $\mathbb{C}$, we can think of the unit circle $\Sph^1$ as a subset of the complex plane. An angle function on a subset $U \subseteq \Sph^1$ is a continuous function $\theta \colon U \to \mathbb{R}$ such that  $\ee^{\ii \theta(z)} = z$ for all $z \in U$. Show that there exists an angle function on an open subset $U \subseteq \Sph^1$ if and only if $U \neq \Sph^1$. For any such angle function, show that $(U, \theta)$ is a smooth coordinate chart for $\Sph^1$ with its standard smooth structure.
  \begin{proof}
    We prove the existence of an angle function $\theta$ on an open subset $U \subseteq \mathbb{S}^1$ for two cases: $U = \mathbb{S}^1$ and $U \subsetneq \mathbb{S}^1$.
    \begin{itemize}
        \item \textbf{Nonexistence for $U = \mathbb{S}^1$:} \\
        Assume such $\theta \colon \mathbb{S}^1 \to \mathbb{R}$ exists. Define the exponential map:
        $$
        f \colon \mathbb{R} \to \mathbb{S}^1, \quad t \mapsto \ee^{\ii t}.
        $$
        By definition, $\theta$ satisfies $f \circ \theta(z) = z$ for all $z \in \mathbb{S}^1$, implying $f$ is injective. However, $f$ is periodic ($f(t+2\pi)=f(t)$), contradicting injectivity. Thus, $\theta$ cannot exist globally.

        \item \textbf{Existence for $U \subsetneq \mathbb{S}^1$:} \\
        Without loss of generality, assume $U = \mathbb{S}^1 \setminus \{p\}$ where $p = (1,0)$. Restrict $f$ to $(0, 2\pi)$:
        $$
        g \coloneqq f|_{(0, 2\pi)} \colon (0, 2\pi) \to U, \quad t \mapsto \ee^{\ii t}.
        $$
        \begin{itemize}
            \item \textit{Bijectivity:} $g$ is bijective by construction, with each $z \in U$ uniquely corresponding to $t \in (0, 2\pi)$.
            \item \textit{Smoothness:} The Jacobian of $g$ at $t$ is:
            $$
            J(g) = \begin{pmatrix} -\sin t \\ \cos t \end{pmatrix},           
            $$
            which has rank 1 everywhere. By the Constant Rank Theorem, $g$ is a diffeomorphism. Its inverse $\varphi \coloneqq g^{-1}$ defines a local angle function on $U$.
        \end{itemize}
        \item \textbf{Smooth Atlas Construction:} \\
        Let $V = \mathbb{S}^1 \setminus \{q\}$ where $q = (-1,0)$, and define:
        $$
        \psi \colon V \to (-\pi, \pi), \quad \ee^{\ii t} \mapsto t.
        $$
        The transition maps between charts $(U, \varphi)$ and $(V, \psi)$ are: 
        \begin{align*}
            \psi \circ \varphi^{-1}(t) &= 
            \begin{cases}
                t         & t \in (0, \pi), \\
                t - 2\pi  & t \in (\pi, 2\pi),
            \end{cases}
            \\
            \varphi \circ \psi^{-1}(t) &=             
            \begin{cases}
                t         & t \in (0, \pi), \\
                t + 2\pi  & t \in (-\pi, 0).
            \end{cases}
        \end{align*}
        Both are smooth on their domains, confirming $\mathcal{A} = \{(U, \varphi), (V, \psi)\}$ is a smooth atlas for $\mathbb{S}^1$.
    \end{itemize}
\end{proof}
\end{problem}

%1-9
\begin{problem}
  Complex projective $n$-space, denoted by $\mathbb{CP}^n$, is the set of all $1$-dimensional 
  complex-linear subspaces of $\mathbb{C}^{n+1}$, with the quotient topology inherited 
  from the natural projection $\pi \colon \mathbb{C}^{n+1} \setminus \{0\} \to \mathbb{CP}^n$. 
  Show that $\mathbb{CP}^n$ is a compact $2n$-dimensional topological manifold, and show how to give it a 
  smooth structure analogous to the one we constructed for $\mathbb{RP}^n$. (We use the 
  correspondence
  $$ (x^1 + \ii y^1, \dots, x^{n+1} + \ii y^{n+1}) \leftrightarrow (x^1, y^1, \dots, x^{n+1}, y^{n+1}) $$
  to identify $\mathbb{C}^{n+1}$ with $\mathbb{R}^{2n+2}$.)
  \begin{proof}
    The construction of smooth structure are exactly the same as in Example 1.5. Here we only prove $\mathbb{CP}^n$ is Hausdorff and second-countable.
    \begin{itemize}
      \item the quotient map $\pi$ is an open map\\
      Let $U$ be an open subset of $\C^{n+1} \setminus \{0\}$, to prove the quotient map $\pi$ is an open map, it only suffices to prove that $\pi^{-1} \circ \pi (U)$ is open in $\C^{n+1} \setminus \{0\}$. Since
      \[
      \pi^{-1} \circ \pi (U) = \bigcup_{t \in \C^{\times}}tU
      \]
      for any fixed $t \in \C^{\times}$, $tU$ is an open subset, we show that their union $\pi^{-1} \circ \pi (U)$ must be open.
      \item \textbf{Hausdorff property}\\
      Let $[z] = [z_0 , \dots , z_n]$ and $[w] = [w_0 , \dots , w_n]$ be two distinct points in $\mathbb{CP}^n$. Then, $z$ and $w$ are not proportional, i.e., there is no $\lambda \in \C^{\times}$ such that $w = \lambda z$. 

      Define the function $f \colon \C^{n+1} \times \C^{n+1} \to \C$ by
      $$
      f(z, w) = \sum_{i < j} |z_i w_j - z_j w_i|^2.
      $$
      This function is zero if and only if $z$ and $w$ are proportional. Since $[z] \neq [w]$, we have $f(z, w) > 0$. 

      By continuity of $f$, there exist open neighborhoods $A \subset \C^{n+1} \setminus \{0\}$ of $z$ and $B \subset \C^{n+1} \setminus \{0\}$ of $w$ such that $f(a, b) > 0$ for all $a \in A$ and $b \in B$. 

      Let $U = \pi(A)$ and $V = \pi(B)$. Since $\pi$ is an open map, $U$ and $V$ are open in $\mathbb{CP}^n$. Moreover, $U$ and $V$ are disjoint, because if $[a] = [b]$ for some $a \in A$ and $b \in B$, then $f(a, b) = 0$, which contradicts the construction of $A$ and $B$. 

      Hence, $\mathbb{CP}^n$ is Hausdorff.
      \item \textbf{Second-countable}\\
      Since $\C^{n+1} \setminus \{0\}$ is second-countable, and $\pi$ is a continuous open map, the quotient space $\mathbb{CP}^n$ is also second-countable. 
      \item \textbf{The compactness of $\mathbb{CP}^n$}\\
      The compactness of $\mathbb{CP}^n$ follows from the fact that it is the continuous image of the unit sphere $\Sph^{2n+1} \subseteq \C^{n+1}$ under $\pi$. 
    \end{itemize}

  \end{proof}
\end{problem}

%1-10
\begin{problem}
  Let $k$ and $n$ be integers satisfying $0 < k < n$, and let $P, Q \subseteq \mathbb{R}^n$ be the linear subspaces spanned by $(e_1,\dots,e_k)$ and $(e_{k+1},\dots,e_n)$, respectively, where $e_i$ is the $i$th standard basis vector for $\mathbb{R}^n$. For any $k$-dimensional subspace $\subseteq \mathbb{R}^n$ that has trivial intersection with $Q$, show that the coordinate representation $\varphi(S)$ constructed in Example 1.36 is the unique $(n-k) \times k$ matrix $B$ such that $S$ is spanned by the columns of the matrix $\bigl(
    \begin{smallmatrix} I_k \\ B \end{smallmatrix}
    \bigr)$, where $I_k$ denotes the $k \times k$ identity matrix.
    \begin{proof}
      We prove the existence and uniqueness of the coordinate representation $\varphi(S) = B$ for a $k$-dimensional subspace $S \subseteq \mathbb{R}^n$ with $S \cap Q = \{0\}$.
      \begin{itemize}
        \item \textbf{Existence of the matrix representation:} \\
        Consider the projection map $\pi_P\colon S \to P$. We claim $\pi_P$ is an isomorphism:
        \begin{itemize}
          \item \textit{Injectivity}: Suppose $\pi_P(s) = 0$ for some $s \in S$. Then $s$ has the form uniquely:
          \begin{equation*}
            s = \pi_P(s) + \pi_Q(s) = \pi_Q(s) \in Q.
          \end{equation*}
          Since $S \cap Q = \{0\}$ by hypothesis, we must have $s = 0$.
          \item \textit{Surjectivity}: As $\dim S = \dim P = k$ and $\pi_P$ is injective, it is automatically surjective by the rank-nullity theorem.
        \end{itemize}
          Thus $\pi_P$ is a vector space isomorphism between $S$ and $P$. Choose 
          \begin{equation*}
            \{\pi_P^{-1}(e_1), \dots, \pi_P^{-1}(e_k)\}
          \end{equation*}
          for the basis of $S$. Since $\{e_1, e_n\}$ is a basis of $V$, we have
          \begin{equation*}
            \pi_P^{-1}(e_i) = e_i + \sum_{j=k+1}^{n}b_{ij}e_j
          \end{equation*}
          Thus $S$ can be spanned by the columns of the matrix
          $$\begin{pmatrix}
            I_k\\ B
          \end{pmatrix}$$
          under the basis $\{e_1, \dots, e_n\}$ where $B = (b_{ij})$.
          \item \textbf{Uniqueness of the matrix $B$:} \\
          Suppose there exist two $(n-k) \times k$ matrices $B$ and $B'$ such that:
          \begin{equation*}
            \operatorname{span}\left(
              \begin{pmatrix} I_k \\ B \end{pmatrix}
              \right) = \operatorname{span}\left(
              \begin{pmatrix} I_k \\ B' \end{pmatrix}
              \right) = S.
          \end{equation*}
          Then there exists an invertible matrix $C \in \mathbb{R}^{k \times k}$ such that:     
          \begin{equation*}
            \begin{pmatrix} I_k \\ B' \end{pmatrix}
            = 
           \begin{pmatrix} I_k \\ B \end{pmatrix}
            C.
          \end{equation*}
          This matrix equation implies:
          \begin{align*}
            I_k &= I_k C \quad \Rightarrow \quad C = I_k, \\
            B' &= B C = B.
          \end{align*}
          Therefore, $B$ is uniquely determined by $S$.
      \end{itemize}
    \end{proof}
\end{problem}

%1-11
\begin{problem}
  Let $M = \overline{\B^n}$, the closed unit ball in $\mathbb{R}^n$. 
  Show that $M$ is a topological manifold with boundary in which each point in $\Sph^{n-1}$ is a boundary point and each point in $\B^n$ is an interior point. 
  Show how to give it a smooth structure such that every smooth interior chart is a smooth chart for the standard smooth structure on $\B^n$. [Hint: consider the map $\pi \circ\sigma^{-1}\colon \mathbb{R}^n\to\mathbb{R}^n$, 
  where $\sigma\colon \Sph^n\to\mathbb{R}^n$ is the stereographic projection (Problem 1-7) and $\pi$ is a projection from $\mathbb{R}^{n+1}$ to $\mathbb{R}^n$ that omits some coordinate other than the last.]
  \begin{proof}
    We establish that $\overline{\B}^n$ is a smooth manifold with boundary, where $\mathbb{S}^{n-1}$ constitutes the boundary and $\mathbb{B}^n$ the interior, by constructing an explicit smooth structure. (This proof proceeds independently of the hint.)
    \begin{itemize}
      \item \textbf{Topological manifold structure:}
      \begin{itemize}
        \item For $x \in \B^n$: The identity chart $(\B^n, \text{id}_{\B^n})$ suffices.
        \item For $x \in \Sph^{n-1}$: We define charts via coordinate projection:
        \begin{gather*}
          \begin{aligned}
            &U_i^+ = \{x \in \mathbb{R}^n \mid x_i > 0\}, \\
            &V_i^+ = U_i^+ \cap \overline{\B}^n, \\
            &\varphi_i = \pi_i \circ \pi_{n+1}^{-1}: V_i^+ \to \mathbb{H}^{n} \cap \B^n,
          \end{aligned}\\
          \varphi_i(x^1, \dots, x^n) = \pi_i(x^1, \dots, x^n, \sqrt{1-|x|^2}) = (x^1, \dots, \widehat{x^i}, \dots, \sqrt{1-|x|^2})
        \end{gather*}
        where $\pi_{i} \colon \Sph^n \to \mathbb{R}^n$  omits the $i$-th coordinate.
        The collection 
        \begin{equation*}
        \{(V_i^{\pm}, \varphi_i)\}
        \end{equation*}
        forms boundary charts since $\pi_i$ and $\pi_{n+1}$ are both homeomorphic on $V_i^+$.
      \end{itemize}
      \item \textbf{Smooth structure:} 
        \begin{itemize}
          \item The charts $\{(V_i^{\pm}, \varphi_i)\}$ are compatible with each other, since the standard smooth structure of $\Sph^n$ ensures transition maps
            \begin{equation*}
              \varphi_j \circ \varphi_i^{-1} = \pi_j \circ \pi_{n+1}^{-1} \circ \pi_{n+1} \circ \pi_i^{-1} = \pi_j \circ \pi_i^{-1}
            \end{equation*}
            are diffeomorphisms on their domains $\varphi_i(V_i^+ \cap V_j^+)$.
          \item Boundary charts and interior chart are compatible, since the Jacobian of transition map
            \begin{equation*}
              |J(\varphi_i \circ \text{id}^{-1}_{\B^n})| = (-1)^{n-1}\dfrac{x^i}{\sqrt{1-|x|^2}} \neq 0
            \end{equation*}
            on its domain $\B^n \cap V_i^{\pm}$. Thus the smooth atlas
            \begin{equation*}
              \mathcal{A} = \{(V_i^{\pm}, \varphi_i)\} \cup (\B^n, \text{id}_{\B^n})
            \end{equation*}
            yields a smooth structure of $\overline{\B}^n$.
          \end{itemize}
      \item \textbf{Boundary and interior identification:} 
        \begin{itemize}
          \item For $x \in \Sph^{n-1}$, some some boundary chart $(V_i^{\pm}, \varphi_i)$ satisfies 
            \begin{equation*}
              \varphi_i(x) = (x^1, \dots, \widehat{x^i}, \dots, 0) \in \partial \mathbb{H}^n,
            \end{equation*}
            confirming $\mathbb{S}^{n-1} \subseteq \partial \overline{\mathbb{B}}^n$ via Theorem 1.46 (Boundary Invariance).
          \item For $x \in \mathbb{B}^n$, the identity chart maps $x$ to $\mathbb{B}^n \subseteq \mathbb{R}^n$, proving $\mathbb{B}^n \subseteq \operatorname{Int}(\overline{\mathbb{B}}^n)$.
          \item Since $\overline{\mathbb{B}}^n = \mathbb{B}^n \cup \mathbb{S}^{n-1}$, we conclude:
            \begin{equation*}
              \partial \overline{\mathbb{B}}^n = \mathbb{S}^{n-1}, \quad \operatorname{Int}(\overline{\mathbb{B}}^n) = \mathbb{B}^n.
            \end{equation*}
      \end{itemize} 
    \end{itemize}
  \end{proof}
\end{problem}

%1-12
\begin{problem}
  Prove Proposition 1.45 (a product of smooth manifolds together with one smooth manifold with boundary is a smooth manifold with boundary).
  \begin{proof}
  \begin{itemize}
    \item \textbf{Model Space Identification}: 
      First observe that $\mathbb{R}^m \times \mathbb{H}^n \cong \mathbb{H}^{m+n}$ via the diffeomorphism:
      \begin{align*}
        \varphi \colon \mathbb{R}^m \times \mathbb{H}^n &\to \mathbb{H}^{m+n} \\
        (x^1,\ldots,x^m,y^1,\ldots,y^n) &\mapsto (x^1,\ldots,x^m,y^1,\ldots,y^n)
      \end{align*}
      This preserves boundaries since $\varphi(\mathbb{R}^m \times \partial\mathbb{H}^n) = \partial\mathbb{H}^{m+n}$.
      \item \textbf{Chart Construction}:
        Let $M = M_1 \times \cdots \times M_k$ (dim $m = \sum m_i$) and $N$ (dim $n$) with $\partial N \neq \emptyset$.
        \begin{itemize}
          \item \textbf{Interior Charts}: For $(p,q) \in M \times \operatorname{Int}(N)$:
          \begin{enumerate}
            \item Take smooth charts $(U_i,\varphi_i)$ about $p_i \in M_i$ with $\varphi_i \colon U_i \to \mathbb{R}^{m_i}$
            \item Take interior chart $(V,\psi)$ about $q \in N$ with $\psi \colon V \to \mathbb{R}^n$
            \item The product chart is:
              $$
              \left( \prod_{i=1}^k U_i \times V, (\varphi_1,\ldots,\varphi_k,\psi) \right)
              $$
              mapping to $\mathbb{R}^m \times \mathbb{R}^n \subseteq \mathbb{H}^{m+n}$
          \end{enumerate}
          \item \textbf{Boundary Charts}: For $(p,q) \in M \times \partial N$:   
          \begin{enumerate}
            \item Take smooth charts $(U_i,\varphi_i)$ as above
            \item Take boundary chart $(V,\psi)$ with $\psi \colon V \to \mathbb{H}^n$ and $\psi(q) \in \partial\mathbb{H}^n$
            \item The product chart is:
              $$
              \left( \prod_{i=1}^k U_i \times V, (\varphi_1,\ldots,\varphi_k,\psi) \right)
              $$
              mapping to $\mathbb{R}^m \times \mathbb{H}^n \cong \mathbb{H}^{m+n}$ with boundary points precisely when $q \in \partial N$
          \end{enumerate}
        \end{itemize}
      \item \textbf{Chart Compatibility}:
        \begin{itemize}
          \item For two interior charts, the transition map is:
            $$
            (\varphi'_1,\ldots,\varphi'_k,\psi') \circ (\varphi_1,\ldots,\varphi_k,\psi)^{-1} = (\varphi'_1\circ\varphi_1^{-1},\ldots,\varphi'_k\circ\varphi_k^{-1}, \psi'\circ\psi^{-1})
            $$
            which is smooth since each component is smooth.
          \item For boundary charts, the same holds because $\psi'\circ\psi^{-1}$ is smooth as a map between subsets of $\mathbb{H}^n$.
          \item For mixed cases (one interior, one boundary chart), the transition maps are smooth by the boundary compatibility of $N$'s charts.
      \end{itemize}
      \item \textbf{Boundary Characterization}:
        \begin{itemize}
          \item If $(p,q)$ is mapped to $\partial\mathbb{H}^{m+n}$ in some chart, then by Theorem 1.46 it holds in all charts, this occurs precisely when $q \in \partial N$, proving:
            $$
            \partial(M \times N) = M \times \partial N
            $$
          \item The interior is correspondingly $M \times \operatorname{Int}(N)$
        \end{itemize}
  \end{itemize}
  Thus $M \times N$ is a smooth manifold with boundary as claimed. \qedhere
  \end{proof}
\end{problem}